\documentclass[12pt]{article}
\usepackage{fullpage}
\usepackage{hyperref}
\renewcommand{\familydefault}{\sfdefault}
\addtolength{\topmargin}{-0.5in}
\addtolength{\textheight}{1in}
\addtolength{\oddsidemargin}{-0.5in}
\addtolength{\evensidemargin}{-0.5in}
\addtolength{\textwidth}{1in}



\begin{document}

\begin{center}
{\Large \textbf{Prediction Markets!}}\\
Joseph Bergman, Jamie Matthews, and Kevin Schechter\\
CMSC396H -- Fall 2016 -- University of Maryland\\
Advisor: John P. Dickerson
\end{center}

\section*{Project Schedule}

\noindent
\textbf{Pick a Team Name - Oct 13} \\
Pick a catchy team name for our group, currently "Prediction Markets." The team name will provide something for us to go off of for the website name and domain. 

\noindent
\textbf{Pick a Domain - Oct 15} \\
Once we have decided on a team name, we will need to decide on a fitting domain name that we can use for our website.

\noindent
\textbf{Read Existing Research - Oct 18} \\
Before next week's CMSC 396 class, we should read the existing research that Dr. Dickerson has provided on current prediction markets, and the CMU prediction market. This will provide us with background knowledge on how to develop a prediction market and some of the complications involved.

\noindent
\textbf{Write "Related Works Section" - Oct 20} \\
Prior to meeting with Dr. Dickerson on Thursday, we will write up a brief related works section to summarize what we learned. This will guide our discussion with Dr. Dickerson on Thursday, so we can determine if we want to use an existing prediction market, or how we may want to modify it. 

\noindent
\textbf{Pick Our Topics to Predict - Oct 25} \\
In addition to predicting the opening of the Iribe Center, decide on any additional events we would like to predict. In addition, we must decide if participants will have X funds for each event, or X funds to allocate to all events. Perhaps X funds to allocate to all events reflects their certainty in their prediction being correct?  

\noindent
\textbf{Write Up How Our Prediction Market(s) Will Work - Oct 25} \\
Based on our discussion with Dr. Dickerson, write up a high-level summary of how our prediction market will work. If we are considering multiple different models, we will write up each of them and why we are considering them.  

\noindent
\textbf{Develop Our Prediction Market - Nov 8} \\
Based on the research done previously and the decision made on October 25, we will implement our prediction market in our chosen language. This market will later be used in our website and must be compatible.

\noindent
\textbf{Test Our Prediction Market - Nov 11} \\
After meeting with our mentor a few days prior, we will make any necessary adjustments to our prediction market by this day, then test our predictor.

\noindent
\textbf{First Status Report - Nov 11} \\
In one page, explain what we have accomplished, what has changed about our project
and projected timeline, what tasks we still need to do, and what obstacles
are impeding us.

\noindent
\textbf{Deploy on Rails - Nov 18} \\
Create a website for our market on which users can buy and sell using fake money. We will have the opportunity to meet with our mentor the day before to answer any final questions and test our market thus far.

\noindent
\textbf{Develop User Authentication System - Nov 21} \\
After discussing with our mentor, we will pick which user authentication system will best work for our purposes and add this component to the website, so that users may buy and sell. 

\noindent
\textbf{Invite People to Sign Up - Nov 23} \\
We will send our product to various members of the community to participate in buying and selling. This will give users a couple days to buy and sell before the second status report is due.

\noindent
\textbf{Second Status Report - Nov 30} \\
In one page, explain what we have accomplished, what has changed about our project
and projected timeline, what tasks we still need to do, and what obstacles
are impeding us.

\noindent
\textbf{Final Paper - Dec 7} \\
In six to ten pages, we will detial the work we have done over the semester. We
will need to explain the problem we were working to solve, list the steps
we took to solve the problem, and present our results in a way that 
demonstrates their usefulness.

\noindent
\textbf{Group Presentation - Dec 7} \\
We will need to reformat the content of our final paper into a presentation
format. This will involve using PowerPoint to present our information in
a more engaging way.




And also here.

To compile this, you'll need to download and install a Latex distribution.  On a Mac, I'd recommend \url{https://tug.org/mactex/}.  On Windows and Linux, I'd probably recommend \url{http://miktex.org/}.

This is how you cite papers:~\cite{Othman13:Gates} and~\cite{Pennock01:Real}.  The information for those citations can be found in the \texttt{refs.bib} file in this directory.

%%%%%%%%%%%%%%%%%%%%%%%%%%%%%%%%%%%%%%%%%%%%%%%%%%%%%%%%%%%%%%%%%%%%%%%%%%%%%%%%
%%%%%%%%%%%%%%%%%%%%%%%%%%%%%%%%%%%%%%%%%%%%%%%%%%%%%%%%%%%%%%%%%%%%%%%%%%%%%%%%
%%%%%%%%%%%%%%%%%%%%%%%%%%%%%%%%%%%%%%%%%%%%%%%%%%%%%%%%%%%%%%%%%%%%%%%%%%%%%%%%
\bibliographystyle{amsplain}
\bibliography{refs}

\end{document}
